\chapter{Conclusioni}
%domestiche
Negli ultimi anni, a causa della crescita del numero di servizi fruibili in streaming e della popolarit\`a del paradigma cloud computing, fornire una connessione domestica affidabile dal punto di vista prestazionale \'e diventato di fondamentale importanza.
In aggiunta, la popolarit\`a dei dispositivi mobili ha anche causato un repentino cambio nel mezzo trasmissivo utilizzato per la connessione in favore di soluzioni wireless a discapito del mezzo cablato.

Per far fronte a problematiche riguardanti la connettivit\`a Wi-Fi numerosi produttori di dispositivi di rete stanno implementando nei loro prodotti soluzioni atte al monitoraggio delle risorse collegate ad una rete.

Nel secondo capitolo, dopo aver introdotto i metodi di accesso al mezzo trasmissivo wireless e rilevanti studi riguardanti le prestazioni di questi sistemi, sono stati introdotti alcuni dei software utilizzati per il monitoraggio.
Le soluzioni presenti attualmente in letteratura hanno per\`o diverse limitazioni.
In primis la maggior parte dei software disponibili sono a pagamento ed il loro funzionamento \'e ristretto ad apparecchiature professionali, difficilmente trovate in reti locali domestiche.
In aggiunta, questi tipi di soluzioni sono spesso riservate ad un uso per dispositivi dello stesso produttore.
Alcuni software open-source forniscono delle metriche sui dispositivi collegati in rete tramite Wi-Fi ma non forniscono una topologia dettagliata della rete, requisito fondamentale per risolvere eventuali malfunzionamenti.

Per questi motivi, nel successivo capitolo si \'e proposta una soluzione per il monitoraggio di reti wireless, in grado di ricostruire una topologia della rete e fornire valori di bont\`a del segnale per ciascun dispositivo collegato.
Questo risultato \'e stato ottenuto catturando ed analizzando il traffico 802.11 di tutti i dispositivi presenti nelle vicinanze.
Pur non essendo esente da limitazioni, come la necessit\`a di disconnettere l'interfaccia Wi-Fi da eventuali reti, la soluzione riesce, facendo uso di diverse euristiche, a fornire una visione generale delle reti senza fili presenti nelle vicinanze.

Sono state riportate successivamente le prove di validazione effettuate su diverse reti che si differiscono per numero e tipo di dispositivi presenti oltre a topologia.
La soluzione proposta, si \'e dimostrata adatta per essere utilizzata su dispositivi con risorse di memoria e calcolo limitati.
Questo \'e dovuto anche al numero di dispositivi generalmente presenti nelle vicinanze dell'apparecchiatura che effettua la cattura ed analisi del traffico.
Ponendoci come obiettivo il monitoraggio di reti domestiche, il numero dei dispositivi \'e ,naturalmente, contenuto.
Le prove effettuate su piccole reti di cui erano note la topologia hanno riportato come la soluzione proposta sia in grado di ricostruire correttamente la topologia di una rete Wi-Fi anche sotto condizioni di basso traffico.
Per quanto riguarda la bont\`a della connettivit\`a si fornisce una stima basata sulla potenza del segnale ed il rumore di fondo, seguendo gli standard utilizzati dai principali gestori di rete.
Conoscendo quindi tutti i dispositivi connessi in rete, la loro potenza di segnale e  come essi siano connessi all'access point la soluzione proposta riesce a determinare  quale sia l'effettivo nodo affetto da disservizi.
