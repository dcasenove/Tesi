%\raggedbottom
\chapter{Stato dell'arte}
In questo capitolo vengono presentate le motivazioni che hanno portato alla stesura di questo elaborato e le soluzioni per rilevazione di disservizi nella connettivit\`a pi\'u utilizzate.

\section{Motivazione}
Come introdotto nel precedente capitolo, lo scopo di questo lavoro \'e di fornire una soluzione per la rilevazione di disservizi in reti locali domestiche.
Lo studio si \'e focalizzato su questo tipo di infrastrutture poich\`e esse rappresentano la maggioranza delle reti e, molto spesso, la loro configurazione \'e  lasciata ad un utente finale con poche conoscenze del campo.
Questo pu\`o portare a prestazioni poco efficienti per quanto riguarda le connessioni Wi-Fi o all'uso di apparecchiature di scarsa qualit\`a che dimuiscono le velocit\`a di download ed upload dei dispositivi.
In aggiunta, access point vicini, se sullo stesso canale Wi-Fi,potrebbero interferire sulla connessione locale.
Per questo motivo, software come Kismet\cite{kismet}, possono essere utilizzati anche per scegliere un canale non sovrautilizzato oltre a monitorare il traffico Wi-Fi ed i device connessi ad una rete.
Negli ultimi tempi, i produttori di router, stanno cercando di implementare in tutti i loro dispositivi metodi di monitoraggio per rilevare disservizi di reti.
Questo permette agli Internet Service Providers (ISP) di fornire una diagnostica iniziale che circoscriva il problema all'interno o all'esterno della rete.
Nel primo caso, l'utente viene informato della presenza del problema nella sua rete locale e quali dispositivi ne siano affetti, mentre nel secondo caso \'e il fornitore del servizio Internet a dover intervenire sulla propria rete per rimediare al disservizio.
Le soluzioni attuali per il monitoraggio e la gestione di rete presentano infatti diverse problematiche quando applicate a reti piccole locali.
\newpage
%In particolare, sono principalmente ristrette ad apparecchiature di fascia alta, molto lontane da quelle fornite dai provider Internet ai loro utenti.
%Le soluzioni attuali per il monitoraggio e gestione di rete, come ad esempio SNMP\cite{rfc1157}, sono infatti principalmente ristrette ad apparecchiature di fascia alta, molto lontane da quelle fornite dai provider Internet ai loro utenti.
%In altri casi, anche alcune funzionalit\`a come il rilevamento di rete CDP sono proprietarie o funzionanti solo con apparati dello stesso produttore.
In particolare il software \'e:
\begin{itemize}
	\item Ristretto ad apparecchiature di fascia alta.
	\item Non sempre fornito di interoperabilit\`a con software di altri produttori.
	\item Difficile da utilizzare per personale non specializzato.
\end{itemize}

Per questi motivi l'elaborato vuol fornire, dopo aver introdotto le tecniche di rilevazione di disservizi pi\`u comuni, una soluzione che possa essere utilizzata in piccole reti da utenti non esperti e su apparecchiature non professionali.
