%\flushbottom
\raggedbottom
\chapter{Introduzione}

Nell'ultimo decennio, con la crescita in popolarit\`a degli smartphone, si \'e assistito ad un aumento e ad una diversificazione di dispositivi connessi in rete senza precedenti.
Una tipica rete non \'e pi\`u composta semplicemente da qualche PC e server, in caso aziendale, ma anche da telefoni, tablet, smart TV, weareables ed elettrodomestici.
Inoltre non \'e poi cos\`i raro per un utente possedere pi\`u di uno di questi device, aumentando drasticamente il numero di apparecchiature collegate in una singola rete.
Secondo previsioni Cisco, il numero di dispositivi connessi a reti IP sar\`a pari al triplo del numero della popolazione globale entro il 2022\cite{CVI}.
Lo stesso studio riporta un forte cambiamento nel tipo di dispositivi connessi, con dispositivi mobili quali smartphone e tablet, sistemi embedded in TV ed elettrodomestici in costante crescita a discapito dei pi\`u tradizionali PC.
Viene stimato che, entro il prossimo triennio, il 51\% dei dispositivi e connessioni saranno di tipo machine-to-machine, ovvero senza interazione umana, e principalmente costituiti da device IoT.
Data la natura dei dispositivi in crescita nelle reti, si osserva anche un cambiamento nel mezzo trasmissivo in favore della connessione senza fili a discapito della connessione cablata.

In contemporanea all'aumento del numero di dispositivi connessi ad Internet si \'e assistito anche ad un cambio nel paradigma di erogazione di servizi in favore del cloud computing e cloud storage.
Ad esempio, servizi come Google Drive e Dropbox permettono di salvare i propri file in remoto ed accederne tramite connessione ad Internet, diminuendo l'uso di memoria nel dispositivo personale a discapito della necessit\`a di connessione performante.
Allo stesso modo, servizi di streaming come Netflix e Spotify forniscono cataloghi multimediali pressoch\`e infiniti.

In un mondo sempre pi\`u connesso digitalmente diventa quindi fondamentale, per una rete locale, essere in grado di sostenere un traffico di dati elevato ed un numero di dispositivi in costante crescita in modo da poter fornire una buona connettivit\`a per una corretta esperienza d'uso.

Sebbene gran parte del traffico verso questi tipi di servizi venga generato tramite connessioni mobili, quali 3G e 4G, queste non saranno oggetto di discussione.
Il motivo principale risiede nel fatto che la qualit\`a della connessione, in questo caso, dipende quasi interamente dalla bont\`a del segnale ricevuto dalle antenne dell'operatore.
Incidono anche fattori metereologici \cite{inproceedings}, il posizionamento delle apparecchiature sul territorio e la loro rispettiva capillarit\`a.
Fattori secondari di qualit\`a del segnale possono essere invece ricondotti alle antenne del dispositivo mobile che usufruisce della connessione ma, anche in questo caso, esse vengono scelte dal fabbricante e quindi sono fuori dal controllo dell'utente.
I temi affrontati da questo elaborato riguardano la qualit\`a del servizio offerto da reti locali e dei dispositivi collegati ad esse.
Gli sviluppi nel mondo tecnologico precedentemente citati hanno dato vita a diverse sfide per fabbricanti di apparecchiature di rete e personale specializzato del settore.
Ad esempio, l'introduzione della connessione senza fili, richiede particolare attenzione per via della natura delle onde radio.
Gli access point devono essere posizionati in modo strategico all'interno del locale dove si vuole instaurare la connessione, tenendo conto di problemi come l'attenuazione del segnale attraverso mura \cite{6971137} e interferenze causate con altri dispositivi attivi sulla stesse frequenze.
%Posizionamento tramite algoritmi
Una soluzione al primo problema si trova nel posizionare apparecchiature come repeater per estendere il campo di copertura mentre l'utilizzo di software professionale pu\`o essere necessario per la scelta corretta di un canale libero da interferenze.
L'aumento vertiginoso del numero dei dispositivi connessi alle reti aumenta anche l'importanza di riuscire a capirne il tipo ed eventuali servizi offerti prima di iniziare una fase di monitoraggio riguardante il traffico di rete.
Ci sono diverse tecniche in letteratura per questo tipo di analisi, sia di tipo attivo che passivo, che verranno presentate ed approfondite nel prossimo capitolo.
Questo tipo di studio, come vedremo, \'e fondamentale per fornire una corretta analisi dei disservizi di una rete locale poich\`e \'e necessario, prima di tutto, avere un'idea della quantit\`a e del tipo dei dispositivi che si andranno a monitorare.
In un secondo momento verranno poi presentati alcuni strumenti per il monitoraggio effettivo della rete che, anche in questo caso, possono essere divisi in passivo o attivo.
Purtroppo le principali limitazioni di questo tipo di strumenti includono l'implementazione in sole apparecchiature professionali e la difficolt\`a d'uso per personale non specializzato.
Lo studio si \'e quindi incentrato sulla possibilit\`a di implementare tecniche per la rilevazione di disservizi in reti locali tenendo in mente la facilit\`a d'uso e la possibilit\`a di implementazione su apparecchiature a basso costo per reti di piccole dimensioni.
%Si vuole inoltre fornire all'utente una localizzazione del problema in rete e cos

%Modem router moderni

%Topologia
%Utile in ambito aziendale?

%Queste soluzioni, per ora brevemente accennate, sono per\`o principalmente rivolte ad un ambito aziendale dato il costo delle apparecchiature che le implementano e la necessit\`a di personale specializzato in grado di farne uso.


%Cambiamento alle reti locali grazie al wifi
%Posizionamento router,repeater
%Cavo
%Differenze con reti casalinghe
%Device che vanno e vengono

%I temi che verranno affrontati riguardano la qualit\`a del servizio di reti locali e dei dispositivi collegati ad esse.

%Questo fenomeno introduce un'altra serie di problemi legati alla natura delle onde radio come attenuazione del segnale dovuto alle mura di edifici ed interferenze con altri dispositivi che offrono connessione Wi-Fi.

%Fruibilita' tramite internet 
%Non si parla di reti mobili
%PROBLEMA
%SOLUZIONE
%PER CHI NON VA LA SOLUZIONE

%Per garantire un operato corretto della rete, le apparecchiature di rete di fascia alta sono dotate di funzioni di monitoraggio che aiutano l'utente nella rilevazione di disservizi.
%Data la natura tecnica del problema ed il costo di tali apparecchiature l'uso di queste soluzioni \'e per\`o principalmente ristretto a reti aziendali e di grandi dimensioni, dove personale specializzato ha il compito di instaurare e supervisionare la rete.


\section{Obiettivo}

Lo scopo di questo elaborato \'e quello di fornire uno strumento in grado di rilevare eventuali disservizi nella connettivit\`a di piccole reti locali dove le apparecchiature presenti non sono dotate di funzioni di monitoraggio.
La crescita non omogenea di questo tipo di reti, il numero spesso imprevedibile di dispositivi connessi e la moltitudine di servizi che questi offrono, rendono, per\`o, necessaria anche una prima analisi di rete finalizzata a determinare la quantit\`a ed il tipo di device connessi.
Successivamente, con metriche che verranno introdotte nei prossimi capitoli, si procede al monitoraggio di tutti i dispositivi appartenenti alla rete. 
In particolare, in caso di malfunzionamenti, si vuole identificare se questi siano dovuti a problematiche interne alla propria Local Area Network (LAN) o alla Wide Area Network (WAN) del provider Internet.
Per disservizi interni alla LAN, successivamente alla localizzazione del problema, si procede proponendo soluzioni all'utente e identificando tutti i dispositivi il cui servizio \'e degradato.
%L'aumento vertiginoso del numero dei dispositivi connessi alle reti aumenta anche l'importanza di riuscire di riuscire a capirne il tipo ed eventuali servizi offerti prima di iniziare una fase di monitoraggio riguardante il traffico di rete.


\section{Contributo originale}

Durante lo studio iniziale si \'e notata la mancanza di strumenti open-source in grado di fornire una visione topologica dei dispositivi Wi-Fi.
Si \`e quindi sviluppata una libreria in grado di monitorare, tramite ispezione di frame 802.11, il traffico Wi-Fi delle reti circostanti per poi fornirne dati relativi alla potenza del segnale dei dispositivi connessi ed una topologia dettagliata.
Questo passaggio permette il discovery di dispositivi Wi-Fi nella nostra rete locale ed un monitoraggio nel tempo della bont\`a del segnale.
I dettagli implementativi e la relativa validazione sono lasciati ai corrispettivi capitoli dell'elaborato.

\newpage

\section{Struttura della tesi}

La tesi \'e divisa in cinque capitoli di cui si elenca un breve sommario:

\begin{itemize}
	\item Capitolo 1: \textbf{Introduzione}, vengono presentati il problema analizzato e le motivazioni che hanno portato alla stesura di questa tesi.
	\item Capitolo 2: \textbf{Stato dell'arte}, vengono descritte le attuali tecnologie utilizzate per la rilevazione di disservizi nella connettivit\`a di rete.
	\item Capitolo 3: \textbf{Soluzione proposta}, vengono esposte la soluzione proposta e la libreria sviluppata. 
	\item Capitolo 4: \textbf{Validazione}, vengono mostrati i risultati ottenuti al fine di validare la soluzione proposta.
	\item Capitolo 5: \textbf{Conclusione e lavoro futuro}, presentazione delle conclusioni raggiunte ed alcune ipotesi per lavori futuri.
\end{itemize}