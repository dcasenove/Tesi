\documentclass[12pt,a4paper,twoside,openright]{book}
\usepackage[T1]{fontenc}
\usepackage[utf8]{inputenc}
\usepackage[italian]{babel}
\usepackage{hyperref}
\usepackage{makeidx}
\usepackage[nottoc,notlot,notlof]{tocbibind}
\usepackage[autostyle,italian=guillemets]{csquotes}
\usepackage[backend=biber]{biblatex}
 \addbibresource{bibliografia-tesi.bib}
\usepackage{amsmath}
\usepackage{amsfonts}
\usepackage{amssymb}
\usepackage{frontespizio}
\usepackage{fancyhdr}
\usepackage{indentfirst}
\usepackage{bold-extra}

\usepackage{lmodern}

\makeindex

\newenvironment{abstract}% 
	{\cleardoublepage%
		\thispagestyle{empty}%
		 \null \vfill\begin{center}%
			\bfseries \abstractname \end{center}}% 
	{\vfill\null}

\newcommand{\fncyfront}{% 
	\fancyhead[RO]{{\footnotesize\rightmark}} 
	\fancyfoot[RO]{\thepage} \fancyhead[LE]{\footnotesize{\leftmark}} 
	\fancyfoot[LE]{\thepage} 
	\fancyhead[RE,LO]{}
	\fancyfoot[C]{}
	\setlength{\headheight}{14.5pt}
\renewcommand{\headrulewidth}{0.3pt}}

\newcommand{\fncymain}{%
	\fancyhead[RO]{{\bfseries\footnotesize\rightmark}} 
	\fancyfoot[RO]{\bfseries\thepage} 
	\fancyhead[LE]{{\bfseries\footnotesize\leftmark}} 
	\fancyfoot[LE]{\bfseries\thepage}
	\fancyfoot[C]{} 
	\setlength{\headheight}{14.5pt}
\renewcommand{\headrulewidth}{2pt}}


\makeatletter
\def\cleardoublepage{\clearpage\if@twoside \ifodd\c@page \else\hbox{}\thispagestyle{empty}\newpage
\if@twocolumn\hbox{}\newpage\fi\fi\fi} \makeatother

\author{Daniel Casenove}
\title{Rilevazione di disservizi nella connettività di rete}

\begin{document}
\pagestyle{fancy}
\fncyfront
\frontmatter
\begin{frontespizio}
\Margini{2.5cm}{1.5cm}{2.5cm}{1cm}
\Istituzione{Università di Pisa}
%\Facolta{SCIENZE MATEMATICHE, FISICHE E NATURALI}
\Dipartimento{Informatica}
\Corso[Laurea Triennale]{Informatica}
\Titoletto{Tesi di Laurea Triennale}
\Titolo{Rilevazione di disservizi nella connettività di rete}
\Logo[2.5cm]{images/cherubino_pant541}
\Candidato{Daniel Casenove} \Relatore{Luca Deri} 
\Annoaccademico{2018/2019}
\end{frontespizio}

\begin{abstract}
Placeholder 
\end{abstract}
\tableofcontents
\fncymain
\mainmatter

%\flushbottom
\raggedbottom
\chapter{Introduzione}

Nell'ultimo decennio, con la crescita in popolarit\`a degli smartphone, si \'e assistito ad un aumento e ad una diversificazione di dispositivi connessi in rete senza precedenti.
Una tipica rete non \'e pi\`u composta semplicemente da qualche PC e server, in caso aziendale, ma anche da telefoni, tablet, smart TV, weareables ed elettrodomestici.
Inoltre non \'e poi cos\`i raro per un utente possedere pi\`u di uno di questi device, aumentando drasticamente il numero di apparecchiature collegate in una singola rete.
Secondo previsioni Cisco, il numero di dispositivi connessi a reti IP sar\`a pari al triplo del numero della popolazione globale entro il 2022\cite{CVI}.
Lo stesso studio riporta un forte cambiamento nel tipo di dispositivi connessi, con dispositivi mobili quali smartphone e tablet, sistemi embedded in TV ed elettrodomestici in costante crescita a discapito dei pi\`u tradizionali PC.
Viene stimato che, entro il prossimo triennio, il 51\% dei dispositivi e connessioni saranno di tipo machine-to-machine, ovvero senza interazione umana, e principalmente costituiti da device IoT.
Data la natura dei dispositivi in crescita nelle reti, si osserva anche un cambiamento nel mezzo trasmissivo in favore della connessione senza fili a discapito della connessione cablata.

In contemporanea all'aumento del numero di dispositivi connessi ad Internet si \'e assistito anche ad un cambio nel paradigma di erogazione di servizi in favore del cloud computing e cloud storage.
Ad esempio, servizi come Google Drive e Dropbox permettono di salvare i propri file in remoto ed accederne tramite connessione ad Internet, diminuendo l'uso di memoria nel dispositivo personale a discapito della necessit\`a di connessione performante.
Allo stesso modo, servizi di streaming come Netflix e Spotify forniscono cataloghi multimediali pressoch\`e infiniti.

In un mondo sempre pi\`u connesso digitalmente diventa quindi fondamentale, per una rete locale, essere in grado di sostenere un traffico di dati elevato ed un numero di dispositivi in costante crescita in modo da poter fornire una buona connettivit\`a per una corretta esperienza d'uso.

Sebbene gran parte del traffico verso questi tipi di servizi venga generato tramite connessioni mobili, quali 3G e 4G, queste non saranno oggetto di discussione.
Il motivo principale risiede nel fatto che la qualit\`a della connessione, in questo caso, dipende quasi interamente dalla bont\`a del segnale ricevuto dalle antenne dell'operatore.
Incidono anche fattori metereologici \cite{inproceedings}, il posizionamento delle apparecchiature sul territorio e la loro rispettiva capillarit\`a.
Fattori secondari di qualit\`a del segnale possono essere invece ricondotti alle antenne del dispositivo mobile che usufruisce della connessione ma, anche in questo caso, esse vengono scelte dal fabbricante e quindi sono fuori dal controllo dell'utente.
I temi affrontati da questo elaborato riguardano la qualit\`a del servizio offerto da reti locali e dei dispositivi collegati ad esse.
Gli sviluppi nel mondo tecnologico precedentemente citati hanno dato vita a diverse sfide per fabbricanti di apparecchiature di rete e personale specializzato del settore.
Ad esempio, l'introduzione della connessione senza fili, richiede particolare attenzione per via della natura delle onde radio.
Gli access point devono essere posizionati in modo strategico all'interno del locale dove si vuole instaurare la connessione, tenendo conto di problemi come l'attenuazione del segnale attraverso mura \cite{6971137} e interferenze causate con altri dispositivi attivi sulla stesse frequenze.
%Posizionamento tramite algoritmi
Una soluzione al primo problema si trova nel posizionare apparecchiature come repeater per estendere il campo di copertura mentre l'utilizzo di software professionale pu\`o essere necessario per la scelta corretta di un canale libero da interferenze.
L'aumento vertiginoso del numero dei dispositivi connessi alle reti aumenta anche l'importanza di riuscire a capirne il tipo ed eventuali servizi offerti prima di iniziare una fase di monitoraggio riguardante il traffico di rete.
Ci sono diverse tecniche in letteratura per questo tipo di analisi, sia di tipo attivo che passivo, che verranno presentate ed approfondite nel prossimo capitolo.
Questo tipo di studio, come vedremo, \'e fondamentale per fornire una corretta analisi dei disservizi di una rete locale poich\`e \'e necessario, prima di tutto, avere un'idea della quantit\`a e del tipo dei dispositivi che si andranno a monitorare.
In un secondo momento verranno poi presentati alcuni strumenti per il monitoraggio effettivo della rete che, anche in questo caso, possono essere divisi in passivo o attivo.
Purtroppo le principali limitazioni di questo tipo di strumenti includono l'implementazione in sole apparecchiature professionali e la difficolt\`a d'uso per personale non specializzato.
Lo studio si \'e quindi incentrato sulla possibilit\`a di implementare tecniche per la rilevazione di disservizi in reti locali tenendo in mente la facilit\`a d'uso e la possibilit\`a di implementazione su apparecchiature a basso costo per reti di piccole dimensioni.
%Si vuole inoltre fornire all'utente una localizzazione del problema in rete e cos

%Modem router moderni

%Topologia
%Utile in ambito aziendale?

%Queste soluzioni, per ora brevemente accennate, sono per\`o principalmente rivolte ad un ambito aziendale dato il costo delle apparecchiature che le implementano e la necessit\`a di personale specializzato in grado di farne uso.


%Cambiamento alle reti locali grazie al wifi
%Posizionamento router,repeater
%Cavo
%Differenze con reti casalinghe
%Device che vanno e vengono

%I temi che verranno affrontati riguardano la qualit\`a del servizio di reti locali e dei dispositivi collegati ad esse.

%Questo fenomeno introduce un'altra serie di problemi legati alla natura delle onde radio come attenuazione del segnale dovuto alle mura di edifici ed interferenze con altri dispositivi che offrono connessione Wi-Fi.

%Fruibilita' tramite internet 
%Non si parla di reti mobili
%PROBLEMA
%SOLUZIONE
%PER CHI NON VA LA SOLUZIONE

%Per garantire un operato corretto della rete, le apparecchiature di rete di fascia alta sono dotate di funzioni di monitoraggio che aiutano l'utente nella rilevazione di disservizi.
%Data la natura tecnica del problema ed il costo di tali apparecchiature l'uso di queste soluzioni \'e per\`o principalmente ristretto a reti aziendali e di grandi dimensioni, dove personale specializzato ha il compito di instaurare e supervisionare la rete.


\section{Obiettivo}

Lo scopo di questo elaborato \'e quello di fornire uno strumento in grado di rilevare eventuali disservizi nella connettivit\`a di piccole reti locali dove le apparecchiature presenti non sono dotate di funzioni di monitoraggio.
La crescita non omogenea di questo tipo di reti, il numero spesso imprevedibile di dispositivi connessi e la moltitudine di servizi che questi offrono, rendono, per\`o, necessaria anche una prima analisi di rete finalizzata a determinare la quantit\`a ed il tipo di device connessi.
Successivamente, con metriche che verranno introdotte nei prossimi capitoli, si procede al monitoraggio di tutti i dispositivi appartenenti alla rete. 
In particolare, in caso di malfunzionamenti, si vuole identificare se questi siano dovuti a problematiche interne alla propria Local Area Network (LAN) o alla Wide Area Network (WAN) del provider Internet.
Per disservizi interni alla LAN, successivamente alla localizzazione del problema, si procede proponendo soluzioni all'utente e identificando tutti i dispositivi il cui servizio \'e degradato.
%L'aumento vertiginoso del numero dei dispositivi connessi alle reti aumenta anche l'importanza di riuscire di riuscire a capirne il tipo ed eventuali servizi offerti prima di iniziare una fase di monitoraggio riguardante il traffico di rete.


\section{Contributo originale}

Durante lo studio iniziale si \'e notata la mancanza di strumenti open-source in grado di fornire una visione topologica dei dispositivi Wi-Fi.
Si \`e quindi sviluppata una libreria in grado di monitorare, tramite ispezione di frame 802.11, il traffico Wi-Fi delle reti circostanti per poi fornirne dati relativi alla potenza del segnale dei dispositivi connessi ed una topologia dettagliata.
Questo passaggio permette il discovery di dispositivi Wi-Fi nella nostra rete locale ed un monitoraggio nel tempo della bont\`a del segnale.
I dettagli implementativi e la relativa validazione sono lasciati ai corrispettivi capitoli dell'elaborato.

\newpage

\section{Struttura della tesi}

La tesi \'e divisa in cinque capitoli di cui si elenca un breve sommario:

\begin{itemize}
	\item Capitolo 1: \textbf{Introduzione}, vengono presentati il problema analizzato e le motivazioni che hanno portato alla stesura di questa tesi.
	\item Capitolo 2: \textbf{Stato dell'arte}, vengono descritte le attuali tecnologie utilizzate per la rilevazione di disservizi nella connettivit\`a di rete.
	\item Capitolo 3: \textbf{Soluzione proposta}, vengono esposte la soluzione proposta e la libreria sviluppata. 
	\item Capitolo 4: \textbf{Validazione}, vengono mostrati i risultati ottenuti al fine di validare la soluzione proposta.
	\item Capitolo 5: \textbf{Conclusione e lavoro futuro}, presentazione delle conclusioni raggiunte ed alcune ipotesi per lavori futuri.
\end{itemize}
\include{2_motivazione}
\chapter{Soluzione proposta}
In questo capitolo viene presentata la soluzione proposta per la rilevazione di disservizi nella connettivit\`a di reti locali.
Si analizza brevemente l'architettura del software sviluppato per poi fornirne una descrizione dettagliata della sua implementazione e del software utilizzato nello sviluppo.

\section{Architettura}
L'architettura della soluzione proposta \'e principalmente suddivisa in due parti come mostrato in figura: %FIGURA
\begin{itemize}

\item una libreria che implementa un'operazione di ARP \cite{rfc826} scan, utilizzata per fornire una metrica del round-trip time (RTT) ed associare ad indirizzi IPv4 nella rete locale i rispettivi indirizzi MAC.
Bench\`e l'utilizzo di questa libreria sia facoltativo, nei capitoli successivi si mostrer\`a come i risultati ottenuti dall'operazione di ARP scan migliorino l'accuratezza della ricostruzione della topologia Wi-Fi.
\item una libreria che, dopo aver catturato pacchetti di traffico di rete ed ottenuto i risultati dell' ARP scan, analizza i frame ricevuti e li utilizza per ricostruire una topologia della rete Wi-Fi fornendo valori di bont\`a del segnale.
Questa libreria \'e in grado di ricostruire topologie di reti sia in modalit\`a di cattura attiva o da file di cattura.
\end{itemize}

Le operazioni di cattura di pacchetti in entrambe le librerie sviluppate sono effettuate utilizzando l'API Pcap \cite{pcap}.
In particolare, nella libreria di ARP scanning vengono catturati frame di tipo Ethernet II mentre nella libreria per la ricostruzione della topologia di una rete i frame catturati sono di tipo 802.11.
%Perche'

\begin{figure}[!htb]
	\centering
	\includegraphics{images/img4.pdf}
	\caption{Architettura soluzione proposta}
	\label{fig:solproposta}
\end{figure}


\section{Implementazione}
Si presentano brevemente, prima dei dettagli implementativi della libreria per la ricostruzione topologica della rete, il funzionamento della libreria utilizzata per la cattura dei pacchetti Pcap e quella implementata per effettuare ARP scanning.
\subsection{Pcap}

Sviluppata da Tcpdump, libpcap  \'e una libreria scritta in linguaggio C che permette la cattura ed il filtraggio di pacchetti di rete.
La scelta di questa libreria si \'e basata su tre principali aspetti:
\begin{itemize}
	\item Facilit\`a d'uso: libpcap offre astrazioni ad alto livello per la cattura ed il filtraggio di pacchetti di rete.
	\item Filtraggio dei pacchetti: mediante la compilazione di un filter, \'e possibile catturare solo i pacchetti di rete di interesse per l'applicazione.  
	\item Compatibilit\`a: la libreria, oltre ad essere disponibile per la maggior parte dei sistemi operativi moderni, presenta numerosi wrapper per essere integrata in linguaggi di programmazione diversi dal C.
\end{itemize}
Di seguito si descrivono i passi necessari ad effettuare una cattura utilizzando libpcap:
\begin{enumerate}
	\item Scelta dell'interfaccia da utilizzare per la cattura.
	\item Inizializzazione dell'interfaccia e dei parametri della sessione.
	\item Creazione e compilazione del filtro da utilizzare per catturare i pacchetti desiderati. 
	\item Ciclo di ascolto in cui vengono analizzati, uno ad uno, i pacchetti catturati.
	\item Chiusura della sessione di cattura.
\end{enumerate}

%\newpage

\subsection{ARP Scan}

La libreria di ARP scan sviluppata permette la scoperta di tutti i dispositivi all'interno della rete locale.
Questa operazione viene effettuata inviando un numero di pacchetti ARP ad ogni possibile indirizzo IPv4 presente nella sottorete per poi, utilizzando libpcap, riceverne eventuali riscontri.

Il risultato ottenuto \'e un'associazione tra indirizzo IP ed indirizzo media access control (MAC), un indirizzo fisico univoco assegnato ad ogni scheda di rete dal proprio produttore.
L'indirizzo MAC cos\`i ottenuto verr\`a poi utilizzato per fornire una maggiore accuratezza nella ricostruzione della topologia Wi-Fi della rete locale in esame.

Sebbene questo sia il motivo principale di implementazione della libreria, \'e inoltre possibile ricavare una misura del round-trip time verso ogni dispositivo connesso alla rete, in modo analogo al ping attraverso richieste ICMP.
A differenza di quest ultimo che pu\`o essere ignorato da alcuni tipi di dispositivi, utilizzando l'ARP ping con un numero di pacchetti appropriato \'e possibile ricevere riscontro da tutti i dispositivi attualmente connessi alla rete locale.
In particolare, i dispositivi che tendono ad ignorare questo tipo di richieste sono quelli di tipo mobile come smartphone e tablet nei momenti di non utilizzo da parte dell'utente.

Bench\`e ci siano gi\`a diverse implementazioni funzionanti per i sistemi operativi pi\`u utilizzati, si \'e deciso di sviluppare una libreria basilare che svolga solo le operazioni necessarie per la ricostruzione topologica della rete.
Si \'e cercato  in questo modo di limitare l'utilizzo di risorse e fornire una soluzione la cui implementazione \'e indipendente dal sistema operativo e basata solamente sull'uso di libpcap.

\newpage

%La libreria \'e stata scritta utilizzando il linguaggio di programmazione C++  

%ARP
%Pcap
%monitor mode, svantaggi
%Tipi di messaggi che filtriamo
%MAC header
%Tipi op e tipi header
%Euristica

%ascolto Attivo / ascolto passivo
\subsection{WiFi-Topology}
Dopo aver introdotto nelle sezioni precedenti una vista dell'architettura, purch\`e basilare, e le librerie fondamentali per il funzionamento della soluzione proposta si discute ora l'implementazione della principale libreria sviluppata durante il tirocinio.

La ricostruzione della topologia Wi-Fi delle reti viene effettuata attraverso l'analisi del traffico originato dalle stazioni presenti nelle vicinanze del dispositivo su cui il software \'e in esecuzione.

Questo tipo di cattura, come descritto precedemente, \'e effettuata utilizzando la libreria libpcap che permette di utilizzare un'interfaccia di tipo Wi-Fi in una particolare modalit\`a, chiamata monitor mode, di cui si espone di seguito il funzionamento.

\subsubsection{Monitor mode}
La cattura in monitor mode, o RFMON (Radio Frequency MONitor), permette ad una interfaccia di rete wireless di catturare tutto il traffico passante per un canale Wi-Fi.

In questa modalit\`a una scheda di rete non \'e associata ad alcun access point o rete ad-hoc e si pone in uno stato di ascolto in maniera completamente trasparente ad altri dispositivi wireless presenti nelle vicinanze.
Per effettuare ci\`o non vengono rispettati i normali comportamenti di una stazione operante con il protocollo 802.11, come ad esempio l'invio di ACK come descritto nel secondo capitolo.
Di conseguenza, in questa modalit\`a, il dispositivo perde la possibilit\`a di trasmettere dati ed il suo utilizzo \'e ristretto ad un singolo canale wireless.
Un'altra limitazione in questo tipo di cattura riguarda il mancato controllo di errori nei pacchetti catturati, effettuato normalmente con un controllo di ridondanza ciclico (CRC).

Nonostante gli svantaggi elencati, questo tipo di modalit\`a trova molto utilizzo nella progrettazione di reti wireless, ad esempio per la scelta di un canale poco utilizzato al fine di diminuire interferenze tra stazioni, o nel cracking di reti protette con WEP .

Nello sviluppo della libreria questa modalit\`a \'e stata utilizzata per catturare diversi tipi di frame 802.11 trasmessi dai dispositivi nelle vicinanze.
Analizzando questo tipo di dati ed i valori di potenze di segnale forniti dal radiotap header \'e stato poi sviluppato un algoritmo per la ricostruzione della topologia delle reti Wi-Fi.
Per fornire una corretta analisi il controllo di ridondanza ciclico per i frame ricevuti  \'e stato integrato nella libreria sviluppata, in modo da poter evitare eventuali incoerenze tra la topologia ricostruita e quella effettiva.
In aggiunta, poich\`e la modalit\`a monitor dissocia la scheda di rete da un access point e cattura tutto il traffico passante per un canale, la libreria sviluppata non si limita a ricostruire una particolare rete di cui si \'e interessati ma fornisce una visione di tutte quelle nelle vicinanze.
Per sopperire all'impossibilit\`a di ascoltare su pi\`u di un canale si \'e fatto uso di uno script per effettuare channel hopping, in modo da poter catturare traffico per qualche secondo su ciascun canale.

Il compromesso principale dell'uso della monitor mode resta quello di dover dedicare completamente un'interfaccia Wi-Fi alla cattura dei frame dissociandola dalla rete locale che si vuole monitorare.
Per questo motivo, su dispositivi dotati di una sola scheda di rete, non \'e possibile mantenere costantemente attiva la cattura perdendo quindi la possibilit\`a di assistere a cambi nella topologia di rete in tempo reale. 
Un recente studio \cite{zanetti2010non} ha evidenziato come sia possibile virtualizzare, senza perdita di prestazioni, interfacce per la cattura di traffico in modalit\`a promiscua o, come in questo caso, in monitor mode.

Nella prossime sezioni vengono introdotti in dettaglio i frame catturati in questa modalit\`a e come essi sono utilizzati per la ricostruzione della topologia della rete.

\subsubsection{MAC Frame}

Prima di mostrare come avviene l'analisi del traffico si introduce la struttura generale dei frame catturati tramite monitor mode ed i tipi rilevanti al funzionamento della libreria.
Nei protocolli di rete wireless 802.11 un MAC frame, rappresentato in figura \ref{fig:macframe}, \'e composto da campi comuni a tutti i tipi di frame e da campi specifici ad alcuni di essi.

\begin{figure}[!htb]
	\centering
	\includegraphics{images/img5.pdf}
	\caption{MAC frame}
	\label{fig:macframe}
\end{figure}

Un frame 802.11 contiene quindi un MAC header di lunghezza pari a 34 byte, un body di lunghezza variabile in base al tipo di frame catturato ed infine un campo di 4 byte per il controllo degli errori.

Di seguito si fornisce una breve descrizione di tutti i campi presenti nel MAC header, in particolare di quelli utilizzati durante l'implementazione della libreria:

\newpage

\begin{itemize}
	\item Frame Control: 2 byte che forniscono informazioni sul tipo del frame.
	\item Duration/ID: 2 byte che indicano alle stazioni la durata della trasmessione e viene usato per inizializzare il valore del NAV introdotto nel secondo capitolo.
	\item Address 1-2-3-4: 6 ottetti che identificano unicamente un dispositivo tramite   indirizzo MAC.
	\item Sequence Control: diviso in due campi di 12 e 4 bit che indicano, rispettivamente, il numero di sequenza ed il numero del frammento del pacchetto.
	\item QoS Control: 2 byte che identificano i parametri QoS in un frame di dati.
	\item HT Control: 2 byte aggiunti dallo standard 802.11n.
	\item Frame Body: campo di lunghezza e tipo variabile, payload del frame.
	\item FCS: 4 byte di frame check sequence, un codice di rilevazione di errore.
\end{itemize}

I campi interessanti per la ricostruzione della topologia di una rete includono il frame control, gli indirizzi MAC, il body del frame ed il FCS.
In particolare, nella figura \ref{fig:framecontrolfields}, possiamo osservare come il frame control sia suddiviso in 11 sottocampi.

\begin{figure}[!htb]
	\centering
	\includegraphics{images/img6.pdf}
	\caption{Frame Control Fields}
	\label{fig:framecontrolfields}
\end{figure}

Il primo campo, protocol version, indica la versione del protocollo 802.11 in uso dal frame ed \'e sempre pari a 0 poich\`e attualmente esiste una sola versione di questo protocollo.

Il secondo e terzo campo del frame control indicano invece il tipo e sottotipo del frame ricevuto.Questo tipo di informazione \'e fondamentale per una corretta analisi poich\`e a diversi tipi e sottotipi di frame corrispondono diversi campi nel frame body.

Data la lunghezza pari a 2 bit, un frame nello standard 802.11 pu\`o essere suddiviso in quattro categorie di tipo diverse, che a loro volta possono contenere sedici sottocategorie codificate con 4 bit:
\begin{itemize}
	\item 00- Management Frame: forniscono informazioni sullo stato della rete e sono utilizzati per la connessione e disconnessione di dispositivi.
	\item 01- Control Frame: assistono la trasmissione di data frame e per amministrare l'accesso al mezzo al mezzo trasmissivo.
	\item 10- Data Frame: contengono dati di protocolli di livello superiore all'interno del loro body.
	\item 11- Reserved: tipo di frame riservato e non utilizzato nello standard 802.11.
\end{itemize}

L'utilizzo e il tipo di analisi effettuata su questi tipi di frame ed i loro sottotipi verr\`a introdotto in apposite sezioni.

I successivi due campi del frame control, To DS e From DS, sono di particolare importanza per lo studio effettuato sulla ricostruzione della topologia di rete.
Questi due bit possono essere utilizzati per determinare quando un frame \'e immesso nel mezzo trasmissivo wireless e quando, invece, ne esce.

Di seguito si evidenziano i possibili valori di verit\`a dei due campi:

\begin{itemize}
	\item To DS=0, From DS=0 : il frame non deve lasciare il mezzo trasmissivo, valore generalmente associato a tipi di frame come: management e control.
	\item To DS=0, From DS=1 : il frame proviene da un access point e sta entrando nel mezzo trasmissivo wireless.
	\item To DS=1, From DS=0 : il frame proviene da un client e sta uscendo dal mezzo trasmissivo wireless.
	\item To DS=1, From DS=1 : il frame \'e destinato ad un'altra rete wireless.
\end{itemize}

Accoppiando questo tipo di analisi del mezzo trasmissivo con i valori di segnale ottenuti tramite Radiotap \'e possibile rilevare anche eventuali dispositivi connessi ad un access point via cavo, purch\`e il traffico analizzato contenga un cambio di mezzo trasmissivo.

Un esempio, che sar\`a discusso anche nel prossimo capitolo, potrebbe essere quello di un router collegato via cavo ad un access point ed un numero di dispositivi connessi in Wi-Fi a quest'ultimo.

\newpage

\subsubsection{MAC Address}

Un MAC address \'e un identificatore unico associato ad un'interfaccia di rete dal proprio costruttore e viene utilizzato nel protocollo 802.11 per l'instradamento dei frame.
L'indirizzo \'e formato da una struttura di 48 bit divisa in 6 ottetti, come mostrato in figura \ref{fig:macaddress}.
%Guidelines for Use of Extended Unique Identifier (EUI), Organizationally Unique Identifier (OUI), and Company ID
Per mantenere l'unicit\`tra tutte le schede di rete prodotte \'e stato introdotto uno standard, chiamato EUI-48 e gestito dalla IEEE, che divide l'indirizzo MAC in due parti:
\begin{itemize}
	\item Organisationally Unique Identifier (OUI): identifica unicamente un produttore di schede di rete ed assegnato dalla IEEE.
	\item Network Interface Controller (NIC): identifica unicamente una determinata scheda di rete e viene assegnata dal prodottore.
\end{itemize}

\begin{figure}[!htb]
	\centering
	\includegraphics{images/img7.pdf}
	\caption{Mac Address}
	\label{fig:macaddress}
\end{figure}

In aggiunta, un indirizzo MAC pu\`o essere universalmente (UAA) o localmente (LAA) assegnato.
Questa diversificazione viene effettuata attraverso il valore del secondo bit meno signficativo del primo ottetto.
Per quanto riguarda gli indirizzi UAA, questo bit viene posto a 0 ed il valore del primo ottetto \'e pari a quello del OUI.
Al contrario, un valore del bit pari ad 1 identifica un indirizzo LAA che viene generalmente assegnato da eventuali amministratori di rete.

La differenziazione del tipo di trasmissione in unicast e multicast avviene in maniera simile, in questo caso l'identificazione avviene mediante il bit meno significativo del primo ottetto.

Un valore pari a 0 equivale ad una trasmissione unicast mentre un valore pari ad 1 una trasmissione di tipo multicast.
Infine, l'indirizzo di broadcast \'e ottenuto ponendo un valore pari ad 1 ad ogni bit dell'indirizzo MAC.

Come visto precedentemente, un MAC header \'e costituito da quattro indirizzi MAC il cui valore e significato varia in base al tipo di frame ed ai valori To DS e From DS.
Gli indirizzi possono essere dei seguenti tipi:
\begin{itemize}
	\item Receiver Address (RA): indirizzo della stazione che riceve il frame.
	\item Transmitter Address (TA): indirizzo della stazione che trasmette il frame.
	\item Basic Service Set Identifier (BSSID): indirizzo dell'access point della rete.
	\item Destination Address (DA): indirizzo di destinazione finale del frame.
	\item Source Address (SA): indirizzo sorgente del frame.
\end{itemize}

Nella figura \ref{table:addressvalue} si mostrano le varie combinazioni che gli indirizzi MAC possono rappresentare in base ai valori presenti nel frame control.

\begin{table}[h]
\centering
\begin{tabular}{| l | l | l | l | l | l |}
	\hline 
	To DS  & From DS & Address 1 & Address 2 & Address 3 & Address 4\\ \hline
	 0 & 0 & RA=DA & TA=SA & BSSID & N/A \\ \hline
     0 & 1 & RA=DA & TA=BSSID & SA & N/A\\ \hline
	 1 & 0 & RA=BSSID & TA=SA & DA & N/A\\ \hline
	 1 & 1 & RA & TA & DA & SA\\ \hline
\end{tabular}
%\end{flushright}
\centering
\caption{Tabella uso indirizzi MAC }
\label{table:addressvalue}
\end{table}

%Aggiungere ref
%Accennare qui indirizzo locale repeater?

\newpage

\subsubsection{Management Frame}
\subsubsection{Control Frame}
\subsubsection{Data Frame}
\subsubsection{Frames utilizzati}
\subsubsection{Analisi ed euristica}

\chapter{Stato dell'arte}
Placeholder
\include{5_validazione}
\chapter{Conclusioni}
%domestiche
Negli ultimi anni, a causa della crescita del numero di servizi fruibili in streaming e della popolarit\`a del paradigma cloud computing, fornire una connessione domestica affidabile dal punto di vista prestazionale \'e diventato di fondamentale importanza.
In aggiunta, la popolarit\`a dei dispositivi mobili ha anche causato un repentino cambio nel mezzo trasmissivo utilizzato per la connessione in favore di soluzioni wireless a discapito del mezzo cablato.

Per far fronte a problematiche riguardanti la connettivit\`a Wi-Fi numerosi produttori di dispositivi di rete stanno implementando nei loro prodotti soluzioni atte al monitoraggio delle risorse collegate ad una rete.

Nel secondo capitolo, dopo aver introdotto i metodi di accesso al mezzo trasmissivo wireless e rilevanti studi riguardanti le prestazioni di questi sistemi, sono stati introdotti alcuni dei software utilizzati per il monitoraggio.
Le soluzioni presenti attualmente in letteratura hanno per\`o diverse limitazioni.
In primis la maggior parte dei software disponibili sono a pagamento ed il loro funzionamento \'e ristretto ad apparecchiature professionali, difficilmente trovate in reti locali domestiche.
In aggiunta, questi tipi di soluzioni sono spesso riservate ad un uso per dispositivi dello stesso produttore.
Alcuni software open-source forniscono delle metriche sui dispositivi collegati in rete tramite Wi-Fi ma non forniscono una topologia dettagliata della rete, requisito fondamentale per risolvere eventuali malfunzionamenti.

Per questi motivi, nel successivo capitolo si \'e proposta una soluzione per il monitoraggio di reti wireless, in grado di ricostruire una topologia della rete e fornire valori di bont\`a del segnale per ciascun dispositivo collegato.
Questo risultato \'e stato ottenuto catturando ed analizzando il traffico 802.11 di tutti i dispositivi presenti nelle vicinanze.
Pur non essendo esente da limitazioni, come la necessit\`a di disconnettere l'interfaccia Wi-Fi da eventuali reti, la soluzione riesce, facendo uso di diverse euristiche, a fornire una visione generale delle reti senza fili presenti nelle vicinanze.

Sono state riportate successivamente le prove di validazione effettuate su diverse reti che si differiscono per numero e tipo di dispositivi presenti oltre a topologia.
La soluzione proposta, si \'e dimostrata adatta per essere utilizzata su dispositivi con risorse di memoria e calcolo limitati.
Questo \'e dovuto anche al numero di dispositivi generalmente presenti nelle vicinanze dell'apparecchiatura che effettua la cattura ed analisi del traffico.
Ponendoci come obiettivo il monitoraggio di reti domestiche, il numero dei dispositivi \'e ,naturalmente, contenuto.
Le prove effettuate su piccole reti di cui erano note la topologia hanno riportato come la soluzione proposta sia in grado di ricostruire correttamente la topologia di una rete Wi-Fi anche sotto condizioni di basso traffico.
Per quanto riguarda la bont\`a della connettivit\`a si fornisce una stima basata sulla potenza del segnale ed il rumore di fondo, seguendo gli standard utilizzati dai principali gestori di rete.
Conoscendo quindi tutti i dispositivi connessi in rete, la loro potenza di segnale e  come essi siano connessi all'access point la soluzione proposta riesce a determinare  quale sia l'effettivo nodo affetto da disservizi.

\include{7_lavoro_futuro}

\cite{placeholder}
\backmatter

\printbibliography[heading=bibintoc]


\end{document}